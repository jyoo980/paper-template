\section{Introduction}
\label{sec:intro}

\todo{
This paper outline gives the general structure of a conference paper about
a new practical verification tool, in the style that is standard for a
software engineering conference such as ICSE or ESEC/FSE. Your paper
may have more sections (or fewer) depending on its topic, but any
SE paper will always have:
\begin{itemize}
\item an introduction, which lays out the problem to be solved, the new technique
used to solve the problem, and brags about how successful we were
(\cref{sec:intro});\footnote{The introduction should always end with a list
of contributions structured like this one.}
\item a description of the technique (\cref{sec:technique});
\item an evaluation section, whose results justify the claims in the
introduction (\cref{sec:evaluation});
\item a limitations or threats to validity section that explains in what
ways the experiments might be misleading, and what we've done to mitigate
those threats (\cref{sec:limitations}); and
\item a related work section that places the work in context (\cref{sec:relatedwork}).
\item a conclusion that recaps the contributions of the paper.
\item often, a short ``data availability'' that indicates whether and where
  your data and experimental scripts are available.
\end{itemize}
}

