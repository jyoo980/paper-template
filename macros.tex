% Use a macro like this for your tool's name. Never actually use your tool's name
% in the text: always use the macro.
\newcommand{\tool}{the Type Inference Framework\xspace}
\newcommand{\Tool}{The Type Inference Framework\xspace}
\newcommand{\toolShort}{TIF\xspace}
%% Mike thinks it's better to use a more readable macro name, such as "\theIndexChecker" rather than "\tool".
%% Martin doesn't really care either way; you can create a macro for your tool's name or leave it as-is.

%%% Todo comments
\newcommand{\todo}[1]{{\color{red}\bfseries [[#1]]}}
%% Comment or uncomment this line.
%\renewcommand{\todo}[1]{\relax}

\newcommand{\manu}[1]{\todo{#1 --MS}}

% Don't show todo commands if this macro is defined.
\ifdefined\notodocomments
  \renewcommand{\todo}[1]{\relax}
\fi

% Use like: \ifanonymous{ANONYMOUS TEXT}\else{NON-ANONNYMOUS TEXT}\fi
% where the "\else{NON-ANONNYMOUS TEXT}" may be omitted.
\newif\ifanonymous
%% Comment or uncomment this line
\anonymoustrue

\newcommand{\anonurl}[1]{\ifanonymous URL removed for anonymity.\else\url{#1}\fi}
\newcommand{\footnoteanonurl}[1]{\footnote{\anonurl{#1}}}

% \|name| or \mathid{name} denotes identifiers and slots in formulas
\def\|#1|{\mathid{#1}}
\newcommand{\mathid}[1]{\ensuremath{\mathit{#1}}}
% \<name> or \codeid{name} denotes computer code identifiers
\def\<#1>{\codeid{#1}}
% \protected\def\codeid#1{\ifmmode{\mbox{\sf{#1}}}\else{\sf #1}\fi}
% \protected\def\codeid#1{\ifmmode{\mbox{\ttfamily{#1}}}\else{\ttfamily #1}\fi}
\protected\def\codeid#1{\ifmmode{\mbox{\smaller\ttfamily{#1}}}\else{\smaller\ttfamily #1}\fi}

\newcommand{\CalledMethodsBottom}{\<@Call\-ed\-Meth\-ods\-Bottom>\xspace}
\newcommand{\CalledMethods}{\<@Call\-ed\-Meth\-ods>\xspace}
\newcommand{\EnsuresCalledMethods}{\<@En\-sures\-Call\-ed\-Meth\-ods>\xspace}
\newcommand{\MustCall}{\codeid{@Must\-Call}\xspace}
\newcommand{\MustCallAlias}{\codeid{@Must\-Call\-Alias}\xspace}
\newcommand{\MustCallUnknown}{\codeid{@Must\-Call\-Unknown}\xspace}
\newcommand{\CreatesMustCallFor}{\<@Creates\-Must\-Call\-For>\xspace}
% Deprecated
\newcommand{\ResetMustCall}{\CreatesMustCallFor}

% "trule" stands for ``type rule''
\newcommand{\trule}[2]{\[\frac{#1}{#2}\]}
\newcommand{\truleinline}[2]{\ensuremath{#1\mathrel{\vdash}#2}}
\newcommand{\hastype}[1]{\mathbin{:}\trtext{#1}}
\newcommand{\trcode}[1]{\codeid{\smaller\smaller #1}}
\newcommand{\trtext}[1]{\mbox{\smaller\smaller #1}}
\newcommand{\trquoted}[1]{\trcode{"}#1\trcode{"}}


%%% Computed values

% For any number that's referenced in the text itself (and a table), create a macro like these rather than copy-pasting.
% These examples are from the WPI paper; you can delete them.

\newcommand{\numTypeSystems}{11\xspace} % Formatter, index, interning, lock, nullness, regex, resourceleak, signature, signedness, InitializedFields, Optional
\newcommand{\numModifiedTypeSystems}{2\xspace} % Formatter, Nullness (not Called Methods, because the postcondition code is general)
\newcommand{\numProjects}{12\xspace}
\newcommand{\numLOC}{88,680\xspace}
\newcommand{\numHumanAnnos}{803\xspace}
\newcommand{\percentInferred}{39\todo{check}\%\xspace}
\newcommand{\warningReductionPercent}{45\todo{check}\%\xspace}
\newcommand{\tsSpecificLoC}{61\todo{check}\xspace}

%%% Miscellaneous

\hyphenation{type-state}        % LaTeX defaults to "types-tate"

%%% Space-saving hacks

% Reduce indentation in lists.
\setlength{\leftmargini}{.75\leftmargini}
\setlength{\leftmarginii}{.75\leftmarginii}
\setlength{\leftmarginiii}{.75\leftmarginiii}

\newcommand{\prefigcaption}{\vspace{-5pt}}
\newcommand{\posttablecaption}{\vspace{-5pt}}

% Reduce the separation between figures and text.
\addtolength{\textfloatsep}{-.25\textfloatsep}
\addtolength{\dbltextfloatsep}{-.25\dbltextfloatsep}
\addtolength{\floatsep}{-.25\floatsep}
\addtolength{\dblfloatsep}{-.25\dblfloatsep}

\newcommand{\zph}{\phantom{0}}
\newcommand{\zzph}{\phantom{00}}

\newcommand{\ie}{i.e.,\xspace}
\newcommand{\eg}{e.g.,\xspace}
